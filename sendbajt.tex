\documentclass[a4paper,polish,titlepage,12pt]{article}
\usepackage[T1]{fontenc}
\usepackage[polish]{babel}
\usepackage{polski}
\usepackage{textcomp}
\usepackage[utf8]{inputenc}
\usepackage{fancyheadings}
\usepackage[final]{graphicx}
\usepackage{pslatex}
\pagestyle{fancy}
\lhead{}
\usepackage{times}
\usepackage{hyperref}
\usepackage{xcolor,listings}
\usepackage{array}
%\usepackage{color}
\usepackage{textcomp}
\lstset{upquote=true}
%\usepackage{graphicx}
\usepackage{lmodern}
\usepackage{wrapfig}
\usepackage[margin=1.3in]{geometry}
\usepackage{csquotes}
\hypersetup{
    colorlinks=true, %set true if you want colored links
%    linktoc=all,     %set to all if you want both sections and subsections linked
    linkcolor=blue,  %choose some color if you want links to stand out
    colorlinks,
    citecolor=blue,
    filecolor=blue,
    urlcolor=blue
}

\begin{document}
\begin{titlepage}
\begin{center}
{\Huge\bf Sendbajt}\\
\vspace{2cm}
{\small Copyright 1998 by <oscyloskop>}
\end{center}
\end{titlepage}

Jest to opowieść o najbardziej chyba spektakularnej grupie hackerskiej w historii Netu oraz najgenialniejszym hackerze wsrod seniorow. Pan Jan S. , bo o nim mowa zaczal sie co prawda interesowac komputerami dopiero w wieku 68 lat lecz efekty jego zainteresowan przerosly jego nasmielsze oczekiwania.

Ale zacznijmy od poczatku.\\
-- Pamietam jak dzis. To bylo w 1987 roku... bylem wlasnie wtedy na poczcie po swoja emeryture (u pani Halinki z 3 okienka) kiedy po raz pierwszy zobaczylem komputer. Stal na biurku przykryty pokrowcem -- wspomina pan Jan.

Wtedy jeszcze nic nie wskazywalo na to ze komputery stana sie zyciowym hobby pana Jana.
-- W roku 1989 w bibliotece wojewodzkiej, gdzie czesto zagladalem tez byly juz komputery... pamietam, ze po raz pierwszy (i ostatni) usiadlem wtedy przed klawiatura. Kompletnie nie wiedzialem co mam zrobic, wiec najpierw przeczytalem cztery razy to co bylo napisane na monitorze. Potem jakos juz poszlo... Tego samego dnia wypozyczylem ksiazke o rosyjskich maszynach cyfrowych z ktorej dowiedzialem sie co to jest bit i bajt oraz kto to jest Lenin.

Od tej pory zaczal sie intesywny okres w zyciu pana S. Cale dnie spedzal w czytelni pochlanialac ksiazki o komputerach, systemach operacyjnych, sieciach komputerowych , ale nie tylko. W kregu zainteresowan pana Jana znalazla sie rowniez telefonia i budowa modemow, co zreszta zaowocowalo w pozniejszym czasie rewolucyjnymi metodami stosowanymi przez grupe "Sendbajt" Ale nie uprzedzajmy faktow. Na poczatku 1989 roku poznal pan S. niejakiego Mieczyslawa R., rowniez emeryta, ktory wiekszosc zycia spedzil na instalowaniu sieci telefonicznych oraz pracy na Strowgerze. Mietek (jak o nim mawial pan Jan) byl wtedy zgorzknialym 64 letnim emerytem, dysponowal jednak duza wiedza praktyczna i dlatego wlasnie pan Jan postanowil zawrzec z nim spolke w celu wyciagniecia od niego mozliwie duzo wiedzy (byc moze juz wtedy istnialy w glowie Jana S. zarysy szatanskiego planu ktory pozniej przyniosl slawe jemu oraz grupie "Sendbajt").

Kolejny rok pan Jan spedzil razem z Mietkiem na dalszym intensywnym szkoleniu w bibliotekach i nie tylko. Prenumerata "Bajtka" otorzyla mu oczy na wiele zagadnien o komputerach o ktorych dotad nie wiedzial nic. Nieodzownym elementem zycia staly sie nocne rozmowy z Mietkiem przy kubku kakao, w czasie ktorych prowadzili ozywione duskusje a to o plikach, a to o sytemach Dos, Unix a to o protokolach sieciowych lub modemowych. Czasami w domu pana Jana pojawiala sie takze pani Bozenka - zona pana Mietka. Przygotowywala im kakao i przysluchiwala sie o czym rozmawiaja. Czasem tez zadawala pytania, ktore jednak nie zawsze mialy sens.

Gdzies tak w sierpniu 1990 pan Jan stworzyl swoj pierwszy program - byl to generator liczb losowych totolotka napisany w basicu commodore 64. Niestety program istnial tylko na kartce z notesu, a to z tego prostego powodu, iz pana Jana nie bylo stac nawet na najtanszy komputer 8 bitowy. Pozniej przyszla nauka assemblera. Okazalo sie ze pan S. ma do tego nadzwyczajny talent. Juz po miesiacu nauczyl sie wszystkich rozkazow procesora 8086. Po kolejnych 3 miesiacach zmudnej nauki mial opanowane wszystkie przerwania i byl w stanie pisac i debuggowac programy w assemblerze i to jedynie za pomoca kilku kartek papieru kancelaryjnego i olowka z gumka.Kiedy pan Jan dowiedzial sie juz dostatecznie duzo o komputerach i systemach operacyjnych przyszla pora na gruntowne studiowanie sieci, zwlaszcza rozleglych. W ciagu pol roku intensywnego wkuwania polaczonego z cwiczeniami praktycznymi pan Jan zdobyl tak wiele informacji, ze mogl np. zakodowac dowolny tekst na ciag znakow ASCII po czym zamienic to na ciag zer i jedynek oraz podzielic na pakiety wyposazone w sume kontrolna, bity stopu, parzystosci i takie tam. Po wielu treningach okazalo sie ze potrafi on z pamieci podac 1 kB plik binarny (ewentualnie zaszyfrowac go np. metoda xor w czasie rzeczywistym). Pan Mietek takze nie proznowal - na polecenie pana Jana zbudowal specjalny aparat telefoniczny z dwoma mikrofonami oraz z trzema sluchawkami.

Mowi pan Jan:
-- Tak pod koniec roku 1991 mialem juz duzo wiadomosci i wiedzialem dokladnie czego chce. Chcialem dostepu do niezliczonych zasobow wiedzy zgromadzonej na wszystkich komputerach swiata. - Mowiac to pan Jan wzrusza sie bardzo i widac, ze silnie to przezywa.
- Przelomem byl styczen 1992. Czytalem wlasnie o najnowszych metodach modulacji sygnalu w pasmie telefonicznym, kiedy wpadl Mietek z nowym "Bajtkiem". Byla tam opublikowana lista wszystkich BBS-ow w Polsce. Postanowilismy sprawdzic te numery. Pozniewaz ja nie mam telefonu, ubralem sie cieplo i poszlismy nie opodal do automatu. Wg "Bajtka" w naszym miescie byly 3 BBSy. Drzacymi rekoma wykrecilem numer pierwszego BBSu i w chwili gdy chcialem wrzucic zeton Mietek powstrzymal mnie i sam energicznie przywalil w aparat centralnie od frontu. Spojarzalem na niego ze zdziwieniem, ale nie bylo czasu na wyjasnienia gdyz w tym momencie nastapilo polaczenie, a ja w sluchawce uslyszalem dzikie piski o duzym natezeniu wpadajace wprost do mego ucha. W pierszym odruchu wypuscilem sluchawke z reki, ale zaraz sie opamietalem i Mietek podal mi sluchawke znowu. Tym razem bylem juz przygotowany i staralem sie rozroznic poszczegolne dzwieki. W mlodosci bylem miedzy innymi muzykiem jazzowym, wiec od razu wylapalem czestotliwosc nosna na 1200 Hz. Slychac bylo regularne sekwencje piskow. Powtorzylo sie to w sumie szesc razy i modem po drugiej stronie sie wylaczyl. Czasu nie bylo duzo, ale juz po tym pierwszym polaczeniu zorientowalem sie, ze mam do czynienia z jakims modemem 2400 a takze rozpoznalem rodzaj modulacji. Za chwile wykrecilismy ten sam numer raz jeszcze i tym razem sprobowalem nawiazac lacznosc. Gwizdanie do mikrofonu niewiele pomoglo, wiec wpadlem na pomysl zeby Mietek wymawial "aaaaaaaa" na czestotliwosci ok 2400 hz a ja w tym czasie wydawalem odpowiednie piski w celu przeprowadzenia handshake'u oraz uzyskania polaczenia z komputerem odleglym z predkoscia przynajmniej 300 bps. Probowalismy jakies 4 razy zanim sie to udalo. Jednak po odebraniu wiadomosci wstepnych oraz zalogowaniu sie do BBSa jako anonymous polaczenie zostalo przerwane, poniewaz Mietek zaniosl sie nagle straszliwym kaszlem. Mnie zreszta tez rozbolalo gardlo od wydawania piskow, oraz reka od notowania zer i jedynek. Prace takze utrudnial fakt, ze musialem jednoczesnie nadawac i deszyfrowac dane. Nalezalo zdecydowanie opuscic budke i udac sie do domu w celu obmyslenia innej strategii, zwlaszcza, ze wokolo zebral sie tlumek mlodych osob przygladajacych sie nam dosc dziwnie. No coz, to moje pierwsze polaczenie z modemem bylo moze niezbyt udane ale za to wiele sie nauczylem.

Co robil pan Jan S. w nastepnych dniach? Otoz zdal on sobie sprawe ze w pojedynke z Mietkiem wiele nie zdzialaja. Potrzebowali pomocy fachowcow. Na pierwszy ogien poszla pani Bozenka, ktora jako regularna bywalczyni coniedzielnej mszy swietej dysponowala odpowim glosem z ktorym pan S. wiazal duze nadzieje.\\
-- Pani Bozenko, pani bedzie pelnila w naszej grupie funkcje generatora fali nosnej.\\
-- Lo Jezu! A co to jest? W imie ojca!\\
-- Spokojnie pani Bozenko, to nic trudnegom niech no pani powie 'aaa'.\\
-- Aa\\
-- Ale tak dlugo "aaaaaa" i tutaj, do mikrofonu prosze.\\
-- Aaaaaaaaaaaaaaaaaaaaaaaaaaaaaaaaaaaaaaa\\
-- Dobrze. No widzi pani? Trudne? Nietrudne. Panie Mietku, odczytal pan czestotliwosc na oscyloskopie?\\
-- Niewiarygodne! Dokladnie 2400 Hz panie Janie!\\
-- Fantastycznie! Jest pani najstarszym generatorem fali nosnych telefonicznych na swiecie.\\
-- No wie pan?\\
-- Zartowalem, he he.\\
-- Panie Janie a jak bedzie sie nazywala nasza grupa?\\
-- Juz to przemyslalem proponuje "Sendbajt". Moze byc ?\\
-- Eee. Dobra.\\

W kolejnych dniach pan Jan pokazywal pani Bozence jak ma sie zachowywac generator fali nosnej, zwlaszcza w przypadku renegocjacji polaczenia oraz zaklocen na linii. Pan Mietek przechodzil intensywny kurs HTMLa (oczywiscie w wersji zerojedynkowej). Po tygodniu do grupy "Sendbajt" dolaczyla jeszcze pani Wanda - dobra znajoma pani Bozenki, ktora wg niej spiewala najglosniej i najpiekniej w calym kosciele.

- Bardzo dobrze! - ucieszyl sie pan Jan - pani bedzie naszym nadajnikiem oraz modulatorem!
- Ale nic nie wiem! Nie umiem! - plakala pani Wanda.
- Jak to nic ? Niech pani powie "pi pi pi pioooupipaupioiopppipipiapappe pi pi"
- pi pi pi pioooupipaupi... jak bylo dalej?
- ...oiopppipipiapappe pi pi ...jeszcze raz!
- pi pi pi pioooupipaupioiopppipipiappe pi pi ...dobrze?
- Opuscila pani jedno pa, ale korekcja bledow modemu odbiorczego powinna sobie z tym poradzic.Poza tym doskonale.Panie Mietku!
- Slucham.
- Prosze przebudowac nasz aparat tak aby drugi mikrofon byl polaczony szeregowo z pierwszym poprzez uklad, ktory pan zaprojektuje tak aby sygnal z drugiego mikrofonu modulowal sygnal pierwszego fazowo, amplitudowo lub czestotliwosciowo w zaleznosci od polozenia przelacznika p3... Druga sluchawka ma miec dodatkowy filtr srodkowoprzepustowy na 1200 Hz ...zreszta tu ma pan wstepny projekt.
- Jasna sprawa, tylko co z tymi krokodylkami, zostaja jak sa ?
- tak, i niech pan skoluje jakies 50 metrow czarnego kabla telefonicznego.
- To sie da zrobic.

Nastepny tydzien uplynal na przygotowaniach. Pan Jan zarywal noce symulujac na kartce mala siec ethernet na szesc komputerow. Bawil sie kopiujac pliki miedzy stanowiskami lub uruchamiajac programy na serwerze. Zabawa ta kosztowala go co prawda dwie ryzy papieru do kserokopiarek ale jego wiedza o dzialaniu sieci wzrosla niepomiernie.

- Nasza pierwsza akcja? No coz, to bylo w piwnicy naszego bloku. Okolo godziny 23:00 zaopatrzeni w latarki, hackomat (jak nazwalismy nasz przyrzad) oraz koszyk na ziemniaki i torbe na kompoty zeszlismy do piwnicy. Mietek od razu odszukal skrzynke z napisem >TP< i wyjal z torby pek kluczy. Po chwili nasz hackomat byl na lini i mielismy dialtone.Wg planu najpierw wykrecilem numer do naszego znajomego BBSu. Panie zajely miejsca przy mikrofonach, Mietek przylozyl swoja sluchawke do ucha, ja swoja i przygotowalem papier i kredki (olowki mi sie juz wtedy skonczyly) Pierwsza proba zalogowania sie nie powiodla, poniewaz pani Wanda z wrazenia krzyknela do mikrofonu i zdalny modem nas rozlaczyl. Jednak za drugim razem udalo sie doprowadzic do polaczenia, co prawda tylko 120 bps, ale jak na poczatek to chyba i tak niezle. Mietek szybko zalapal o co chodzi pozniej juz sam odbieral i deszyfrowal wiadomosci. Dzieki temu ja moglem sie zajac przetwarzaniem danych. Naprawde byloby z nami krucho, gdybu nie to, ze Mietek pozyczyl od swojego syna kalkulator. To byl taki prosty kalkulator, ale mial co trzeba, tzn dodawanie i mnozenie. Kiedy juz sie zalogowalem do systemu pierwsza rzecza jaka zrobilem bylo przejecie praw menedzera BBSu wg mojej metody obmyslonej z pol roku wczesniej. Nie spodziewalem sie ze pojdzie az tak latwo. Niestety po 15 minutach polaczenia pani Bozenka nie wytrzymala i powiedziala ze nie moze dluzej krzyczec "aaaa", ze ona tez chce byc procesorem i inne takie bzdury. Przez nia zerwalismy takie swietnie zapowiadajace sie haczenie. Ale nic to. Zdazylem i tak skasowac wiekszosc plikow systemowych. Kiedy Mietek doprowadzil swoja zone do porzadku i moglismy juz kontynuowac postanowilismy sprobowac czegos innego.Polaczylismy sie z serwerem dosyc duzej firmy L*** z naszego miasta. Okazalo sie ze maja aktywne konto guest. Nic prostszego. Po wejsciu do systemu w ciagu 5 minut zdobylem uprawnienia roota i ku mojej nieopisanej radosci okazalo sie ze serwer ma lacze z inernetem. Niedowierzajac sprawdzilem cala kartke obliczen czy sie nie pomylilem czasem przy dodawaniu liczb ujemnych w systemie osemkowym, bo z tym mialem zawsze troche klopotu. No ale wszystko sie potwierdzilo. Zakrylem mikrofon reka i krzyknalem do Mietka: Udalo sie! Jestesmy w Internecie! niestety nasze panie wytwarzaly taki zgielk, ze prawdopodobnie mnie i tak nie usluszal. Ale ja juz bylem tam gdzie chcialem byc zawsze. Pierwsze co zrobilem do polaczylem sie z serwerem firmy Seagate Technologies (znalem dobrze ich system operacyjny z jednej ksiazki) i wlamalem sie na strone WWW. Nie tracac czasu przekazalem paleczke naszemu specowi od HTMLa , czyli panu Mietkowi, sam zas zajalem jego miejsce. Tak jak sie umowilismy wczesniej Mietek dokonal zmian bezposrednio w kodzie html przy pomocy edytora dysku na serwerze. Teraz trudne zadanie czekalo pania Wande. Musiala nadawac przez 20 minut tekst naszego manifestu...

Kolejne miesiace plynely grupie "sendbajt" szybko. Po pierwszych sukcesach na stronach WWW probowali wlamywania na amerykanskie serwery wojskowe i rzadowe, co bylo od zawsze skrytym marzeniem Jana S. Niestety, pomimo poprawienia (na skutek zaprawy czlonkow "Sendbajt") parametrow transmisji (dochodzila ona do 1200 bps ) nie dalo sie w dalszym ciagu sciagac wiekszych plikow binarnych. Rekordem grupy byl download kodu zrodlowego do Internet Explorera v 2.0 (po wlamaniu na serwer firmy Microsoft). Poprawilo to nawigacje w sieci www gdyz pan Mietek nauczyl sie tego kodu na pamiec i robil po prostu za przegladarke (jak bylo trzeba to szkicowal na kartce jpgi i gify zeby kazdy mogl podziwiac szate graficzna danej strony). Tymczasem pan Jan zaliczal coraz to nowe miejsca www, haczyl i ewentualnie niszczyl serwery internetowe jeden za drugim. Jednym slowem grupa rozwijala sie i z dnia na dzien stawala sie w Sieci coraz bardziej popularna. Na wszystkich administratorow padl blady strach. Wiekszosc z nich zaczela do wymiany informacji uzywac tradycyjnej poczty snail-mail, do tego stopnia byli sterroryzowani przez czlonkow grupy "Sendbajt". Oczywiscie przez caly ten czas grupa korzystala podczas uprawiania swego procederu z roznych numerow telefononow, poczatkowo sasiadow z bloku, ale pozniej pan Mietek wynalazl swietne miejsce kolo przedszkola dwie ulice dalej. Chodzili tam wiec nocami, wpinali hackomat i siadali w krzakach, z daleka od ludzi.

Pewnie sie spodziewacie, ze w koncu policja nakryla grupe "Sendbajt" i zirytowani admini ukamienowali za miastem jej czlonkow, wzglednie Jan S. wyladowal w wiezieniu jak przystalo na hackera-legende ? Otoz nie. Dzialalnosc grupy trwalaby zapewne po dzis dzien gdyby pan Jan nie odkryl nowej pasjii zyciowej - mianowicie wedkarstwa. No niestety bez pana Jana grupa "Sendbajt" szybko sie rozpadla. Spotykaja sie jednak czasem w piwnicy jak za starych czasow i przesiaduja na IRCu lub pan Jan sciaga sobie stronki o wedkarstwie. Poza tym sa szczesliwi. Admini tez, ze cala sprawa przycichla... Nadal wydaje im sie ze ich systemy sa dobrze zabezpieczone i moga spac spokojnie. Niech spia... 

\end{document}
