\documentclass[a4paper,polish,titlepage,12pt]{article}
\usepackage[T1]{fontenc}
\usepackage[polish]{babel}
\usepackage{polski}
\usepackage{textcomp}
\usepackage[utf8]{inputenc}
\usepackage{fancyheadings}
\usepackage[final]{graphicx}
\usepackage{pslatex}
\pagestyle{fancy}
\lhead{}
\usepackage{times}
\usepackage{hyperref}
\usepackage{xcolor,listings}
\usepackage{array}
%\usepackage{color}
\usepackage{textcomp}
\lstset{upquote=true}
%\usepackage{graphicx}
\usepackage{lmodern}
\usepackage{wrapfig}
\usepackage[margin=1.3in]{geometry}
\usepackage{csquotes}
\hypersetup{
    colorlinks=true, %set true if you want colored links
%    linktoc=all,     %set to all if you want both sections and subsections linked
    linkcolor=blue,  %choose some color if you want links to stand out
    colorlinks,
    citecolor=blue,
    filecolor=blue,
    urlcolor=blue
}

\begin{document}
\begin{titlepage}
\begin{center}
{\Huge\bf Sendbajt}\\
\vspace{2cm}
{\small Copyright 1998 by <oscyloskop>}
\end{center}
\end{titlepage}

Jest to opowieść o najbardziej chyba spektakularnej grupie hackerskiej w historii Netu oraz najgenialniejszym hackerze wsrod seniorów. Pan Jan S., bo o nim mowa zaczął się co prawda interesować komputerami dopiero w wieku 68 lat, lecz efekty jego zainteresowań przerosły jego najśmielsze oczekiwania.

Ale zacznijmy od początku.\\
-- Pamiętam jak dziś. To bylo w 1987 roku... byłem właśnie wtedy na poczcie po swoją emeryturę (u pani Halinki z 3 okienka) kiedy po raz pierwszy zobaczyłem komputer. Stał na biurku przykryty pokrowcem -- wspomina pan Jan.

Wtedy jeszcze nic nie wskazywało na to że komputery staną się życiowym hobby pana Jana.
-- W roku 1989 w bibliotece wojewódzkiej, gdzie często zaglądałem też były już komputery... pamiętam, że po raz pierwszy (i ostatni) usiadłem wtedy przed klawiaturą. Kompletnie nie wiedziałem co mam zrobić, więc najpierw przeczytalem cztery razy to co było napisane na monitorze. Potem jakoś już poszło... Tego samego dnia wypożyczyłem książke o rosyjskich maszynach cyfrowych z której dowiedziałem się co to jest bit i bajt oraz kto to jest Lenin.

Od tej pory zaczął się intensywny okres w życiu pana S. Całe dnie spędzał w czytelni pochłaniając książki o komputerach, systemach operacyjnych, sieciach komputerowych, ale nie tylko.
W kregu zainteresowań pana Jana znalazła się rownież telefonia i budowa modemów, co zresztą zaowocowało w późniejszym czasie rewolucyjnymi metodami stosowanymi przez grupę
,,Sendbajt'' Ale nie uprzedzajmy faktów. Na poczatku 1989 roku poznal pan S. niejakiego Mieczysława R., również emeryta, który większość życia spędził na instalowaniu sieci
telefonicznych oraz pracy na Strowgerze. Mietek (jak o nim mawial pan Jan) był wtedy zgorzkniałym 64 letnim emerytem, dysponował jednak duża wiedzą praktyczną i dlatego
właśnie pan Jan postanowił zawrzeć z nim spółkę w celu wyciągnięcia od niego możliwie dużo wiedzy (być może już wtedy istniały w głowie Jana S. zarysy szatańskiego planu,
który później przyniósł sławę jemu oraz grupie ,,Sendbajt'').

Kolejny rok pan Jan spędził razem z Mietkiem na dalszym intensywnym szkoleniu w bibliotekach i nie tylko. Prenumerata ,,Bajtka'' otworzyła mu oczy na wiele zagadnień o
komputerach, o których dotąd nie wiedział nic. Nieodzownym elementem życia stały się nocne rozmowy z Mietkiem przy kubku kakao, w czasie których prowadzili ożywione duskusje,
a to o plikach, a to o systemach Dos, Unix, a to o protokołach sieciowych lub modemowych. Czasami w domu pana Jana pojawiała się także pani Bożenka - żona pana Mietka.
Przygotowywała im kakao i przysłuchiwała się o czym rozmawiają. Czasem też zadawała pytania, które jednak nie zawsze mialy sens.

Gdzieś tak w sierpniu 1990 pan Jan stworzył swój pierwszy program - był to generator liczb losowych totolotka napisany w basicu commodore 64. Niestety program istniał tylko
na kartce z notesu, a to z tego prostego powodu, iż pana Jana nie było stać nawet na najtańszy komputer 8 bitowy. Później przyszła nauka assemblera. Okazało się że pan S. ma
do tego nadzwyczajny talent. Już po miesiącu nauczył się wszystkich rozkazów procesora 8086. Po kolejnych 3 miesiącach żmudnej nauki miał opanowane wszystkie przerwania i był
w stanie pisać i debuggować programy w assemblerze i to jedynie za pomocą kilku kartek papieru kancelaryjnego i olówka z gumką. Kiedy pan Jan dowiedział się już dostatecznie
dużo o komputerach i systemach operacyjnych, przyszla pora na gruntowne studiowanie sieci, zwlaszcza rozległych. W ciągu pól roku intensywnego wkuwania połączonego z
ćwiczeniami praktycznymi pan Jan zdobyl tak wiele informacji, że mógł np. zakodować dowolny tekst na ciag znakow \verb+ASCII+ po czym zamienić to na ciąg zer i jedynek oraz
podzielić na pakiety wyposażone w sumę kontrolną, bity stopu, parzystości i takie tam. Po wielu treningach okazało się, że potrafi on z pamieci podac 1 kB plik binarny
(ewentualnie zaszyfrować go np. metodą xor w czasie rzeczywistym). Pan Mietek także nie próżnował - na polecenie pana Jana zbudował specjalny aparat telefoniczny z dwoma
mikrofonami oraz z trzema sluchawkami.

Mówi pan Jan:\\
-- Tak pod koniec roku 1991 miałem już dużo wiadomości i wiedziałem dokładnie czego chcę. Chciałem dostępu do niezliczonych zasobów wiedzy zgromadzonej na wszystkich
komputerach świata. - Mówiąc to pan Jan wzrusza się bardzo i widać, że silnie to przeżywa.\\
-- Przełomem był styczeń 1992. Czytałem właśnie o najnowszych metodach modulacji sygnału w paśmie telefonicznym, kiedy wpadł Mietek z nowym ,,Bajtkiem''. Była tam
opublikowana lista wszystkich BBS--ow w Polsce. Postanowiliśmy sprawdzić te numery. Ponieważ ja nie mam telefonu, ubrałem się ciepło i poszliśmy nieopodal do automatu.
Wg ,,Bajtka'' w naszym mieście byly 3 BBSy. Drżącymi rękoma wykręciłem numer pierwszego BBSu i w chwili gdy chciałem wrzucić żeton Mietek powstrzymał mnie i sam energicznie
przywalił w aparat centralnie od frontu. Spojrzałem na niego ze zdziwieniem, ale nie było czasu na wyjaśnienia, gdyż w tym momencie nastąpiło połączenie, a ja w słuchawce
usłyszałem dzikie piski o dużym natężeniu wpadające wprost do mego ucha. W pierwszym odruchu wypuściłem słuchawkę z ręki, ale zaraz się opamiętałem i Mietek podał mi słuchawkę
 znowu. Tym razem byłem już przygotowany i starałem się rozróżnić poszczególne dźwięki. W młodości byłem między innymi muzykiem jazzowym, więc od razu wyłapałem częstotliwość
nośna na 1200 Hz. Słychać było regularne sekwencje pisków. Powtórzyło się to w sumie sześć razy i modem po drugiej stronie się wyłączył. Czasu nie było dużo, ale już po tym
pierwszym połączeniu zorientowalem się, że mam do czynienia z jakimś modemem 2400, a także rozpoznałem rodzaj modulacji. Za chwilę wykręciliśmy ten sam numer raz jeszcze i
tym razem spróbowałem nawiązać łączność. Gwizdanie do mikrofonu niewiele pomogło, wiec wpadłem na pomysł żeby Mietek wymawiał ,,aaaaaaaa'' na częstotliwości ok 2400 Hz a ja w
tym czasie wydawałem odpowiednie piski w celu przeprowadzenia handshake'u oraz uzyskania połączenia z komputerem odległym z prędkością przynajmniej 300 bps. Próbowaliśmy
jakieś 4 razy zanim się to udało. Jednak po odebraniu wiadomości wstępnych oraz zalogowaniu się do BBSa jako anonymous połączenie zostało przerwane, ponieważ Mietek zaniosł
się nagle straszliwym kaszlem. Mnie zreszta też rozbolało gardło od wydawania pisków, oraz ręka od notowania zer i jedynek. Pracę także utrudniał fakt, że musiałem
jednocześnie nadawać i deszyfrować dane. Należalo zdecydowanie opuścić budkę i udać się do domu w celu obmyślenia innej strategii, zwłaszcza, że wokolo zebrał się tłumek
młodych osób przyglądających się nam dość dziwnie. No coż, to moje pierwsze połączenie z modemem było może niezbyt udane ale za to wiele się nauczylem.

Co robił pan Jan S. w następnych dniach? Otóż zdał on sobie sprawę że w pojedynkę z Mietkiem wiele nie zdziałają. Potrzebowali pomocy fachowców. Na pierwszy ogień poszła
pani Bożenka, która jako regularna bywalczyni coniedzielnej mszy swiętej dysponowała odpowiednim głosem z którym pan S. wiązał duże nadzieje.\\
-- Pani Bożenko, pani będzie pełniła w naszej grupie funkcję generatora fali nośnej.\\
-- Ło Jezu! A co to jest? W imię ojca!\\
-- Spokojnie pani Bożenko, to nic trudnego niech no pani powie ,,aaa''.\\
-- Aa\\
-- Ale tak długo ,,aaaaa'' i tutaj, do mikrofonu proszę.\\
-- Aaaaaaaaaaaaaaaaaaaaaaaaaaaaaaaaaaaaaaa\\
-- Dobrze. No widzi pani? Trudne? Nietrudne. Panie Mietku, odczytał pan częstotliwość na oscyloskopie?\\
-- Niewiarygodne! Dokladnie 2400 Hz panie Janie!\\
-- Fantastycznie! Jest pani najstarszym generatorem fali nośnych telefonicznych na świecie.\\
-- No wie pan?\\
-- Żartowalem, he he.\\
-- Panie Janie, a jak będzie się nazywała nasza grupa?\\
-- Już to przemyślałem; proponuję ,,Sendbajt''. Może być ?\\
-- Eee. Dobra.\\

W kolejnych dniach pan Jan pokazywał pani Bożence jak ma się zachowywać generator fali nośnej, zwłaszcza w przypadku renegocjacji połączenia oraz
zakłóceń na linii. Pan Mietek przechodził intensywny kurs HTMLa (oczywiście w wersji zerojedynkowej). Po tygodniu do grupy ,,Sendbajt'' dołączyła
jeszcze pani Wanda - dobra znajoma pani Bożenki, która wg niej śpiewała najgłośniej i najpiękniej w całym kościele.

-- Bardzo dobrze! -- ucieszył się pan Jan -- pani będzie naszym nadajnikiem oraz modulatorem!\\
-- Ale nic nie wiem! Nie umiem! - płakała pani Wanda.\\
-- Jak to nic ? Niech pani powie ,,pi pi pi pioooupipaupioiopppipipiapappe pi pi''\\
-- pi pi pi pioooupipaupi\dots jak bylo dalej?\\
-- \dots oiopppipipiapappe pi pi\dots jeszcze raz!\\
-- pi pi pi pioooupipaupioiopppipipiappe pi pi\dots dobrze?\\
-- Opuściła pani jedno pa, ale korekcja błędow modemu odbiorczego powinna sobie z tym poradzić. Poza tym doskonale. Panie Mietku!\\
-- Slucham.\\
-- Proszę przebudować nasz aparat tak aby drugi mikrofon był połączony szeregowo z pierwszym poprzez układ, który pan zaprojektuje tak aby sygnał z
drugiego mikrofonu modulowal sygnał pierwszego fazowo, amplitudowo lub częstotliwościowo w zależności od położenia przełącznika p3\dots Druga słuchawka
 ma mieć dodatkowy filtr środkowoprzepustowy na 1200 Hz\dots zreszta tu ma pan wstepny projekt.\\
-- Jasna sprawa, tylko co z tymi krokodylkami, zostają jak są ?\\
-- tak, i niech pan skołuje jakieś 50 metrów czarnego kabla telefonicznego.\\
-- To się da zrobić.

Następny tydzień upłynął na przygotowaniach. Pan Jan zarywał noce symulując na kartce małą sieć ethernet na sześć komputerów. Bawił się kopiując pliki między stanowiskami lub
uruchamiając programy na serwerze. Zabawa ta kosztowala go co prawda dwie ryzy papieru do kserokopiarek, ale jego wiedza o działaniu sieci wzrosła niepomiernie.

-- Nasza pierwsza akcja? No cóż, to było w piwnicy naszego bloku. Około godziny 23:00 zaopatrzeni w latarki, hackomat (jak nazwaliśmy nasz przyrząd) oraz koszyk na ziemniaki
i torbę na kompoty zeszliśmy do piwnicy. Mietek od razu odszukał skrzynke z napisem >TP< i wyjął z torby pęk kluczy. Po chwili nasz hackomat był na lini i mieliśmy dialtone.
Wg planu najpierw wykręciłem numer do naszego znajomego BBSu. Panie zajęły miejsca przy mikrofonach, Mietek przyłożył swoją słuchawkę do ucha, ja swoją i przygotowałem papier
i kredki (olówki mi się już wtedy skończyły). Pierwsza próba zalogowania się nie powiodła, ponieważ pani Wanda z wrażenia krzyknęła do mikrofonu i zdalny modem nas rozlaczyl.
Jednak za drugim razem udało się doprowadzić do połączenia, co prawda tylko 120 bps, ale jak na początek to chyba i tak nieźle. Mietek szybko załapal o co chodzi później już
sam odbierał i deszyfrował wiadomości. Dzięki temu ja mogłem się zająć przetwarzaniem danych. Naprawdę byłoby z nami krucho, gdyby nie to, że Mietek pożyczyl od swojego syna
kalkulator. To był taki prosty kalkulator, ale miał co trzeba, tzn. dodawanie i mnożenie. Kiedy już się zalogowalem do systemu pierwszą rzeczą jaką zrobiłem było przejęcie
praw menedżera BBSu wg mojej metody obmyślonej z pół roku wcześniej. Nie spodziewałem się że pojdzie aż tak łatwo. Niestety po 15 minutach połączenia pani Bożenka nie
wytrzymała i powiedzilaa że nie może dłużej krzyczeć ,,aaaa'', że ona też chce być procesorem i inne takie bzdury. Przez nią zerwaliśmy takie świetnie zapowiadające się
haczenie. Ale nic to. Zdążyłem i tak skasować większość plików systemowych. Kiedy Mietek doprowadził swoją żonę do porządku i mogliśmy już kontynuować, postanowiliśmy
sprobować czegoś innego. Połączyliśmy się z serwerem dosyć dużej firmy L*** z naszego miasta. Okazało się że mają aktywne konto guest. Nic prostszego. Po wejściu do systemu w
ciągu 5 minut zdobyłem uprawnienia root-a i ku mojej nieopisanej radości okazało się że serwer ma łącze z inernetem. Niedowierzając sprawdziłem całą kartkę obliczeń czy się
nie pomyliłem czasem przy dodawaniu liczb ujemnych w systemie ósemkowym, bo z tym miałem zawsze trochę kłopotu. No, ale wszystko się potwierdziło. Zakryłem mikrofon reką i
krzyknąłem do Mietka: Udało się! Jesteśmy w Internecie! Niestety nasze panie wytwarzały taki zgiełk, że prawdopodobnie mnie i tak nie usłyszał. Ale ja już byłem tam gdzie
chciałem być zawsze. Pierwsze co zrobiłem do połączyłem się z serwerem firmy Seagate Technologies (znałem dobrze ich system operacyjny z jednej książki) i włamałem się na 
stronę WWW. Nie tracąc czasu przekazałem pałeczkę naszemu specowi od HTMLa , czyli panu Mietkowi, sam zaś zająłem jego miejsce. Tak jak się umówiliśmy wcześniej, Mietek
dokonał zmian bezpośrednio w kodzie html za pomocą edytora dysku na serwerze. Teraz trudne zadanie czekało panią Wande. Musiała nadawać przez 20 minut tekst naszego manifestu\dots

Kolejne miesiące płynęły grupie ,,Sendbajt'' szybko. Po pierwszych sukcesach na stronach WWW próbowali włamywania na amerykańskie serwery wojskowe i rządowe, co było od
zawsze skrytym marzeniem Jana S. Niestety, pomimo poprawienia (na skutek zaprawy członków ,,Sendbajt'') parametrów transmisji (dochodziła ona do 1200 bps) nie dało się w
dalszym ciągu sciągnąć większych plików binarnych. Rekordem grupy był download kodu źródłowego do Internet Explorera v2.0 (po włamaniu na serwer firmy Microsoft). Poprawiło
to nawigację w sieci WWW gdyż pan Mietek nauczył się tego kodu na pamięć i robił po prostu za przegladarkę (jak było trzeba to szkicowal na kartce jpgi i gify żeby każdy mógł
podziwiać szatę graficznę danej strony). Tymczasem pan Jan zaliczał coraz to nowe miejsca WWW, haczył i ewentualnie niszczył serwery internetowe jeden za drugim. Jednym
słowem grupa rozwijała się i z dnia na dzień stawała się w Sieci coraz bardziej popularna. Na wszystkich administratorów padł blady strach. Większość z nich zaczęła do
wymiany informacji używać tradycyjnej poczty snail-mail, do tego stopnia byli sterroryzowani przez członków grupy ,,Sendbajt''. Oczywiście przez cały ten czas grupa
korzystała podczas uprawiania swego procederu z różnych numerów telefononów, początkowo sąsiadów z bloku, ale później pan Mietek wynalazł świetne miejsce koło przedszkola
dwie ulice dalej. Chodzili tam wiec nocami, wpinali hackomat i siadali w krzakach, z daleka od ludzi.

Pewnie się spodziewacie, że w koncu policja nakryła grupę ,,Sendbajt'' i zirytowani admini ukamienowali za miastem jej czlonków, względnie Jan S. wylądował w więzieniu jak
przystało na hackera-legendę ? Otóż nie. Działalność grupy trwałaby zapewne po dziś dzień gdyby pan Jan nie odkrył nowej pasji życiowej -- mianowicie wędkarstwa. No niestety
bez pana Jana grupa ,,Sendbajt'' szybko się rozpadła. Spotykają się jednak czasem w piwnicy jak za starych czasow i przesiadują na IRCu lub pan Jan sciąga sobie stronki o
wędkarstwie. Poza tym są szczęśliwi. Admini też, że cała sprawa przycichła\dots Nadal wydaje im się że ich systemy są dobrze zabezpieczone i mogą spać spokojnie. Niech śpią\dots

\end{document}
