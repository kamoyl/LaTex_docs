\documentclass[a4paper,polish,titlepage,12pt]{article}
\usepackage[T1]{fontenc}
\usepackage[polish]{babel}
\usepackage{polski}
\usepackage{textcomp}
\usepackage[utf8]{inputenc}
\usepackage{fancyheadings}
\usepackage[final]{graphicx}
\usepackage{pslatex}
\pagestyle{fancy}
\lhead{}
\usepackage{times}
\usepackage{hyperref}
\usepackage{xcolor,listings}
\usepackage{array}
%\usepackage{color}
\usepackage{textcomp}
\lstset{upquote=true}
%\usepackage{graphicx}
\usepackage{lmodern}
\usepackage{wrapfig}
\usepackage[margin=1.3in]{geometry}
\usepackage{csquotes}
\hypersetup{
    colorlinks=true, %set true if you want colored links
%    linktoc=all,     %set to all if you want both sections and subsections linked
    linkcolor=blue,  %choose some color if you want links to stand out
    colorlinks,
    citecolor=blue,
    filecolor=blue,
    urlcolor=blue
}

\begin{document}
\begin{titlepage}
\begin{center}
{\Huge\bf Sendbajt}\\
\vspace{2cm}
{\small Copyright 1998 by <oscyloskop>}
\end{center}
\end{titlepage}

Jest to opowieść o najbardziej chyba spektakularnej grupie hackerskiej w historii Netu oraz najgenialniejszym hackerze wsrod seniorów. Pan Jan S., bo o nim mowa zaczął się co prawda interesować komputerami dopiero w wieku 68 lat, lecz efekty jego zainteresowań przerosły jego najśmielsze oczekiwania.

Ale zacznijmy od początku.\\
-- Pamiętam jak dziś. To bylo w 1987 roku... byłem właśnie wtedy na poczcie po swoją emeryturę (u pani Halinki z 3 okienka) kiedy po raz pierwszy zobaczyłem komputer. Stał na biurku przykryty pokrowcem -- wspomina pan Jan.

Wtedy jeszcze nic nie wskazywało na to że komputery staną się życiowym hobby pana Jana.
-- W roku 1989 w bibliotece wojewódzkiej, gdzie często zaglądałem też były już komputery... pamiętam, że po raz pierwszy (i ostatni) usiadłem wtedy przed klawiaturą. Kompletnie nie wiedziałem co mam zrobić, więc najpierw przeczytalem cztery razy to co było napisane na monitorze. Potem jakoś już poszło... Tego samego dnia wypożyczyłem książke o rosyjskich maszynach cyfrowych z której dowiedziałem się co to jest bit i bajt oraz kto to jest Lenin.

Od tej pory zaczął się intensywny okres w życiu pana S. Całe dnie spędzał w czytelni pochłaniając książki o komputerach, systemach operacyjnych, sieciach komputerowych, ale nie tylko. 
W kregu zainteresowań pana Jana znalazła się rownież telefonia i budowa modemów, co zresztą zaowocowało w późniejszym czasie rewolucyjnymi metodami stosowanymi przez grupę 
,,Sendbajt'' Ale nie uprzedzajmy faktów. Na poczatku 1989 roku poznal pan S. niejakiego Mieczysława R., również emeryta, który większość życia spędził na instalowaniu sieci 
telefonicznych oraz pracy na Strowgerze. Mietek (jak o nim mawial pan Jan) był wtedy zgorzkniałym 64 letnim emerytem, dysponował jednak duża wiedzą praktyczną i dlatego 
właśnie pan Jan postanowił zawrzeć z nim spółkę w celu wyciągnięcia od niego możliwie dużo wiedzy (być może już wtedy istniały w głowie Jana S. zarysy szatańskiego planu, 
który później przyniósł sławę jemu oraz grupie ,,Sendbajt'').

Kolejny rok pan Jan spędził razem z Mietkiem na dalszym intensywnym szkoleniu w bibliotekach i nie tylko. Prenumerata ,,Bajtka'' otworzyła mu oczy na wiele zagadnień o 
komputerach, o których dotąd nie wiedział nic. Nieodzownym elementem życia stały się nocne rozmowy z Mietkiem przy kubku kakao, w czasie których prowadzili ożywione duskusje, 
a to o plikach, a to o systemach Dos, Unix, a to o protokołach sieciowych lub modemowych. Czasami w domu pana Jana pojawiała się także pani Bożenka - żona pana Mietka. 
Przygotowywała im kakao i przysłuchiwała się o czym rozmawiają. Czasem też zadawała pytania, które jednak nie zawsze mialy sens.

Gdzieś tak w sierpniu 1990 pan Jan stworzył swój pierwszy program - był to generator liczb losowych totolotka napisany w basicu commodore 64. Niestety program istniał tylko 
na kartce z notesu, a to z tego prostego powodu, iż pana Jana nie było stać nawet na najtańszy komputer 8 bitowy. Później przyszła nauka assemblera. Okazało się że pan S. ma 
do tego nadzwyczajny talent. Już po miesiącu nauczył się wszystkich rozkazów procesora 8086. Po kolejnych 3 miesiącach żmudnej nauki miał opanowane wszystkie przerwania i był 
w stanie pisać i debuggować programy w assemblerze i to jedynie za pomocą kilku kartek papieru kancelaryjnego i olówka z gumką. Kiedy pan Jan dowiedział się już dostatecznie 
dużo o komputerach i systemach operacyjnych, przyszla pora na gruntowne studiowanie sieci, zwlaszcza rozległych. W ciągu pól roku intensywnego wkuwania połączonego z 
ćwiczeniami praktycznymi pan Jan zdobyl tak wiele informacji, że mógł np. zakodować dowolny tekst na ciag znakow \verb+ASCII+ po czym zamienić to na ciąg zer i jedynek oraz 
podzielić na pakiety wyposażone w sumę kontrolną, bity stopu, parzystości i takie tam. Po wielu treningach okazało się, że potrafi on z pamieci podac 1 kB plik binarny 
(ewentualnie zaszyfrować go np. metodą xor w czasie rzeczywistym). Pan Mietek także nie próżnował - na polecenie pana Jana zbudował specjalny aparat telefoniczny z dwoma 
mikrofonami oraz z trzema sluchawkami.

Mówi pan Jan:\\
-- Tak pod koniec roku 1991 miałem już dużo wiadomości i wiedziałem dokładnie czego chcę. Chciałem dostępu do niezliczonych zasobów wiedzy zgromadzonej na wszystkich 
komputerach świata. - Mówiąc to pan Jan wzrusza się bardzo i widać, że silnie to przeżywa.\\
-- Przełomem był styczeń 1992. Czytałem właśnie o najnowszych metodach modulacji sygnału w paśmie telefonicznym, kiedy wpadł Mietek z nowym ,,Bajtkiem''. Była tam 
opublikowana lista wszystkich BBS--ow w Polsce. Postanowiliśmy sprawdzić te numery. Ponieważ ja nie mam telefonu, ubrałem się ciepło i poszliśmy nieopodal do automatu. 
Wg ,,Bajtka'' w naszym mieście byly 3 BBSy. Drżącymi rękoma wykręciłem numer pierwszego BBSu i w chwili gdy chciałem wrzucić żeton Mietek powstrzymał mnie i sam energicznie 
przywalił w aparat centralnie od frontu. Spojrzałem na niego ze zdziwieniem, ale nie było czasu na wyjaśnienia, gdyż w tym momencie nastąpiło połączenie, a ja w słuchawce 
usłyszałem dzikie piski o dużym natężeniu wpadające wprost do mego ucha. W pierwszym odruchu wypuściłem słuchawkę z ręki, ale zaraz się opamiętałem i Mietek podał mi słuchawkę
 znowu. Tym razem byłem już przygotowany i starałem się rozróżnić poszczególne dźwięki. W młodości byłem między innymi muzykiem jazzowym, więc od razu wyłapałem częstotliwość 
nośna na 1200 Hz. Słychać było regularne sekwencje pisków. Powtórzyło się to w sumie sześć razy i modem po drugiej stronie się wyłączył. Czasu nie było dużo, ale już po tym 
pierwszym połączeniu zorientowalem się, że mam do czynienia z jakimś modemem 2400, a także rozpoznałem rodzaj modulacji. Za chwilę wykręciliśmy ten sam numer raz jeszcze i 
tym razem spróbowałem nawiązać łączność. Gwizdanie do mikrofonu niewiele pomogło, wiec wpadłem na pomysł żeby Mietek wymawiał ,,aaaaaaaa'' na częstotliwości ok 2400 Hz a ja w 
tym czasie wydawałem odpowiednie piski w celu przeprowadzenia handshake'u oraz uzyskania połączenia z komputerem odległym z prędkością przynajmniej 300 bps. Próbowaliśmy 
jakieś 4 razy zanim się to udało. Jednak po odebraniu wiadomości wstępnych oraz zalogowaniu się do BBSa jako anonymous połączenie zostało przerwane, ponieważ Mietek zaniosł 
się nagle straszliwym kaszlem. Mnie zreszta też rozbolało gardło od wydawania pisków, oraz ręka od notowania zer i jedynek. Pracę także utrudniał fakt, że musiałem 
jednocześnie nadawać i deszyfrować dane. Należalo zdecydowanie opuścić budkę i udać się do domu w celu obmyślenia innej strategii, zwłaszcza, że wokolo zebrał się tłumek 
młodych osób przyglądających się nam dość dziwnie. No coż, to moje pierwsze połączenie z modemem było może niezbyt udane ale za to wiele się nauczylem.

Co robił pan Jan S. w następnych dniach? Otóż zdał on sobie sprawę że w pojedynkę z Mietkiem wiele nie zdziałają. Potrzebowali pomocy fachowców. Na pierwszy ogień poszła 
pani Bożenka, która jako regularna bywalczyni coniedzielnej mszy swiętej dysponowała odpowiednim głosem z którym pan S. wiązał duże nadzieje.\\
-- Pani Bożenko, pani będzie pełniła w naszej grupie funkcję generatora fali nośnej.\\
-- Ło Jezu! A co to jest? W imię ojca!\\
-- Spokojnie pani Bożenko, to nic trudnego niech no pani powie ,,aaa''.\\
-- Aa\\
-- Ale tak długo ,,aaaaa'' i tutaj, do mikrofonu proszę.\\
-- Aaaaaaaaaaaaaaaaaaaaaaaaaaaaaaaaaaaaaaa\\
-- Dobrze. No widzi pani? Trudne? Nietrudne. Panie Mietku, odczytał pan częstotliwość na oscyloskopie?\\
-- Niewiarygodne! Dokladnie 2400 Hz panie Janie!\\
-- Fantastycznie! Jest pani najstarszym generatorem fali nośnych telefonicznych na świecie.\\
-- No wie pan?\\
-- Żartowalem, he he.\\
-- Panie Janie, a jak będzie się nazywała nasza grupa?\\
-- Już to przemyślałem; proponuję ,,Sendbajt''. Może być ?\\
-- Eee. Dobra.\\

W kolejnych dniach pan Jan pokazywal pani Bozence jak ma się zachowywac generator fali nosnej, zwlaszcza w przypadku renegocjacji polaczenia oraz zaklocen na linii. Pan Mietek przechodzil intensywny kurs HTMLa (oczywiscie w wersji zerojedynkowej). Po tygodniu do grupy ,,Sendbajt'' dolaczyla jeszcze pani Wanda - dobra znajoma pani Bozenki, ktora wg niej spiewala najglosniej i najpiekniej w calym kosciele.

- Bardzo dobrze! - ucieszyl się pan Jan - pani bedzie naszym nadajnikiem oraz modulatorem!
- Ale nic nie wiem! Nie umiem! - plakala pani Wanda.
- Jak to nic ? Niech pani powie "pi pi pi pioooupipaupioiopppipipiapappe pi pi"
- pi pi pi pioooupipaupi... jak bylo dalej?
- ...oiopppipipiapappe pi pi ...jeszcze raz!
- pi pi pi pioooupipaupioiopppipipiappe pi pi ...dobrze?
- Opuscila pani jedno pa, ale korekcja bledow modemu odbiorczego powinna sobie z tym poradzic.Poza tym doskonale.Panie Mietku!
- Slucham.
- Prosze przebudowac nasz aparat tak aby drugi mikrofon byl polaczony szeregowo z pierwszym poprzez uklad, który pan zaprojektuje tak aby sygnal z drugiego mikrofonu modulowal sygnal pierwszego fazowo, amplitudowo lub czestotliwosciowo w zaleznosci od polozenia przelacznika p3... Druga sluchawka ma miec dodatkowy filtr srodkowoprzepustowy na 1200 Hz ...zreszta tu ma pan wstepny projekt.
- Jasna sprawa, tylko co z tymi krokodylkami, zostaja jak sa ?
- tak, i niech pan skoluje jakies 50 metrow czarnego kabla telefonicznego.
- To się da zrobic.

Nastepny tydzien uplynal na przygotowaniach. Pan Jan zarywal noce symulujac na kartce mala siec ethernet na szesc komputerow. Bawil się kopiujac pliki miedzy stanowiskami lub uruchamiajac programy na serwerze. Zabawa ta kosztowala go co prawda dwie ryzy papieru do kserokopiarek ale jego wiedza o dzialaniu sieci wzrosla niepomiernie.

- Nasza pierwsza akcja? No coz, to bylo w piwnicy naszego bloku. Okolo godziny 23:00 zaopatrzeni w latarki, hackomat (jak nazwalismy nasz przyrzad) oraz koszyk na ziemniaki i torbe na kompoty zeszlismy do piwnicy. Mietek od razu odszukal skrzynke z napisem >TP< i wyjal z torby pek kluczy. Po chwili nasz hackomat byl na lini i mielismy dialtone.Wg planu najpierw wykrecilem numer do naszego znajomego BBSu. Panie zajely miejsca przy mikrofonach, Mietek przylozyl swoja sluchawke do ucha, ja swoja i przygotowalem papier i kredki (olowki mi się już wtedy skonczyly) Pierwsza proba zalogowania się nie powiodla, poniewaz pani Wanda z wrazenia krzyknela do mikrofonu i zdalny modem nas rozlaczyl. Jednak za drugim razem udalo się doprowadzic do polaczenia, co prawda tylko 120 bps, ale jak na poczatek to chyba i tak niezle. Mietek szybko zalapal o co chodzi pozniej już sam odbieral i deszyfrowal wiadomosci. Dzieki temu ja moglem się zajac przetwarzaniem danych. Naprawde byloby z nami krucho, gdybu nie to, że Mietek pozyczyl od swojego syna kalkulator. To byl taki prosty kalkulator, ale mial co trzeba, tzn dodawanie i mnozenie. Kiedy już się zalogowalem do systemu pierwsza rzecza jaka zrobilem bylo przejecie praw menedzera BBSu wg mojej metody obmyslonej z pol roku wczesniej. Nie spodziewalem się że pojdzie az tak latwo. Niestety po 15 minutach polaczenia pani Bożenka nie wytrzymala i powiedziala że nie moze dluzej krzyczec "aaaa", że ona też chce byc procesorem i inne takie bzdury. Przez nia zerwalismy takie swietnie zapowiadajace się haczenie. Ale nic to. Zdazylem i tak skasowac wiekszosc plikow systemowych. Kiedy Mietek doprowadzil swoja zone do porzadku i moglismy już kontynuowac postanowilismy sprobowac czegos innego.Polaczylismy się z serwerem dosyc duzej firmy L*** z naszego miasta. Okazalo się że maja aktywne konto guest. Nic prostszego. Po wejsciu do systemu w ciagu 5 minut zdobylem uprawnienia roota i ku mojej nieopisanej radosci okazalo się że serwer ma lacze z inernetem. Niedowierzajac sprawdzilem cala kartke obliczen czy się nie pomylilem czasem przy dodawaniu liczb ujemnych w systemie osemkowym, bo z tym mialem zawsze troche klopotu. No ale wszystko się potwierdzilo. Zakrylem mikrofon reka i krzyknalem do Mietka: Udalo sie! Jestesmy w Internecie! niestety nasze panie wytwarzaly taki zgielk, że prawdopodobnie mnie i tak nie usluszal. Ale ja już bylem tam gdzie chcialem byc zawsze. Pierwsze co zrobilem do polaczylem się z serwerem firmy Seagate Technologies (znalem dobrze ich system operacyjny z jednej ksiazki) i wlamalem się na strone WWW. Nie tracac czasu przekazalem paleczke naszemu specowi od HTMLa , czyli panu Mietkowi, sam zas zajalem jego miejsce. Tak jak się umowilismy wczesniej Mietek dokonal zmian bezposrednio w kodzie html przy pomocy edytora dysku na serwerze. Teraz trudne zadanie czekalo pania Wande. Musiala nadawac przez 20 minut tekst naszego manifestu...

Kolejne miesiace plynely grupie "sendbajt" szybko. Po pierwszych sukcesach na stronach WWW probowali wlamywania na amerykanskie serwery wojskowe i rzadowe, co bylo od zawsze skrytym marzeniem Jana S. Niestety, pomimo poprawienia (na skutek zaprawy czlonkow ,,Sendbajt'') parametrow transmisji (dochodzila ona do 1200 bps ) nie dalo się w dalszym ciagu sciagac wiekszych plikow binarnych. Rekordem grupy byl download kodu zrodlowego do Internet Explorera v 2.0 (po wlamaniu na serwer firmy Microsoft). Poprawilo to nawigacje w sieci www gdyz pan Mietek nauczyl się tego kodu na pamiec i robil po prostu za przegladarke (jak bylo trzeba to szkicowal na kartce jpgi i gify zeby kazdy mogl podziwiac szate graficzna danej strony). Tymczasem pan Jan zaliczal coraz to nowe miejsca www, haczyl i ewentualnie niszczyl serwery internetowe jeden za drugim. Jednym slowem grupa rozwijala się i z dnia na dzien stawala się w Sieci coraz bardziej popularna. Na wszystkich administratorow padl blady strach. Wiekszosc z nich zaczela do wymiany informacji uzywac tradycyjnej poczty snail-mail, do tego stopnia byli sterroryzowani przez czlonkow grupy ,,Sendbajt''. Oczywiscie przez caly ten czas grupa korzystala podczas uprawiania swego procederu z roznych numerow telefononow, poczatkowo sasiadow z bloku, ale pozniej pan Mietek wynalazl swietne miejsce kolo przedszkola dwie ulice dalej. Chodzili tam wiec nocami, wpinali hackomat i siadali w krzakach, z daleka od ludzi.

Pewnie się spodziewacie, że w koncu policja nakryla grupe ,,Sendbajt'' i zirytowani admini ukamienowali za miastem jej czlonkow, wzglednie Jan S. wyladowal w wiezieniu jak przystalo na hackera-legende ? Otoz nie. Dzialalnosc grupy trwalaby zapewne po dzis dzien gdyby pan Jan nie odkryl nowej pasjii zyciowej - mianowicie wedkarstwa. No niestety bez pana Jana grupa ,,Sendbajt'' szybko się rozpadla. Spotykaja się jednak czasem w piwnicy jak za starych czasow i przesiaduja na IRCu lub pan Jan sciaga sobie stronki o wedkarstwie. Poza tym sa szczesliwi. Admini tez, że cala sprawa przycichla... Nadal wydaje im się że ich systemy sa dobrze zabezpieczone i moga spac spokojnie. Niech spia... 

\end{document}
