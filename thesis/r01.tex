\chapter{Wprowadzenie. Cel pracy}
\label{r01}
Celem pracy jest zaprojektowanie i realizacja systemu do zarządzania wielokomputerowymi
serwerami WWW oraz przedstawienie całokształtu tematyki w zakresie równoważenia obciążeń
systemów wielokomputerowych i problematyki z tym związanej.

Działanie ogólnoświatowej sieci rozległej -- Internetu, rozumiane jest jako ,,luźno zorganizowana
międzynarodowa współpraca autonomicznych, połączonych ze sobą sieci komputerowych, w oparciu o komunikację
host--to--host poprzez dobrowolną przynależność do otwartych protokołów i procedur zdefiniowanych w dokumentach
Internet Standards, RFC 1310, RFC1311 i  RFC 1312''.

Początków Internetu należy szukać w projektach realizowanych na zlecenie Departamentu Obrony USA przez agencję
ARPA (ang. \emph{Advanced Research Projects Agency}) oraz jej następczynię -- agencję DARPA (ang. \emph{Defense Advanced
Research Projects Agency}). Przed tymi instytucjami postawiono zadanie opracowania standardów umożliwiających
budowę rozległej sieci komunikacyjnej odpornej na unicestwienie nawet poprzez atak nuklearny znacznej części
jej węzłów \cite{barylo1}. W wyniku prowadzonych prac w roku 1971 przekazana zostaje do użytku sieć ARPANET. Stosowane w niej
protokoły transmisji nie zapewniały jednak wystarczającej przepustowości dla stale rosnącej liczby użytkowników
i już w roku 1973 pojawiają się propozycje pierwszych protokołów z rodziny TCP/IP -- \emph{de facto} dzisiejszego
standardu transmisji internetowych. W roku 1981 na bazie ARPANET--u utworzona zostaje CSN (ang. \emph{Computer Science
Newtwork})  wykorzystywana wspólnie przez abonentów wojskowych i cywilnych. W roku 1982
protokoły TCP/IP wypierają całkowicie starsze protokoły (np. protokół NCP),  a do CSN średnio co 20 dni
podżączany jest nowy komputer. W roku 1984 następuje podział sieci na część wojskowł -- MILNET i cywilną,
która zachowuje tradycyjną nazwę ARPANET. W miarę jak zaczyna przekraczać ona granice USA i w miarę przybywania
abonentów komercyjnych zaczyna pojawiać się nazwa Internet. W roku 1990 decyzją rządu USA szkielet sieci ARPANET
skżadający się z  czterech węzłów komutacyjnch (Los Angeles, Santa Barbara, Uniwersytet Stanforda i Uniwersytet
Utah) zbudowanych w oparciu o minikomputery Honeywell 316 zostaje uznany za przestarzały i przeznaczony do
rozbiórki. Zastępuje go sieć NSFNET, która do dziś jest szkieletową siecią amerykańskiej części Internetu.

Kluczowż ze względu na treść tej pracy datę w historii Internetu jest rok 1990. Wtedy to w Europejskim Ośrodku
Badań Jżdrowych CERN w Genewie zostaż opracowany i uruchomiony pierwszy serwer WWW (ang. \emph{World Wide Web}). WWW
jest systemem hipertekstowym tzn. przechowujżcym dane w postaci sieci dokumentów (plików) pożżczonych poprzez
odsyżacze. Sam dokument może zawierać tekst, grafikę, dźwięk, animacje,  sekwencje video i inne rodzaje danych.
Elementy wskazywane przez odsyżacze mogż znajdować się w tym samym dokumencie lub być innymi dokumentami na tym
samym lub innym serwerze, co dokument zawierajżcy dany odsyżacz. Obecnie WWW to zbiór dziesiżtków tysięcy,
pożżczonych sieciż odsyżaczy, serwerów w wielu krajach. Każdy serwer udostępnia użytkownikom zgromadzone na nim
dokumenty o różnorodnej treści informacyjnej, reklamowej, rozrywkowej, handlowej (e--banki, sklepy internetowe) itp.
Narzędziem umożliwiajżcym zwykżemu użytkownikowi Internetu korzystanie z zasobów WWW jest przeglżdarka
(ang. \emph{browser}). Komunikacja przeglżdarki z serwerem WWW odbywa się w architekturze klient--serwer i jest
zawsze inicjowana przez zapytania ze strony klienta (przeglżdarki). Odpowiadajżc na zapytania przeglżdarki serwer
WWW przesyża do niej żżdane dokumenty z wykorzystaniem protokożu HTTP (ang.
\emph{Hyper Text Transfer Protocol}). Najczęściej stosowanym językiem opisu dokumentu hipertekstowego jest HTML
(ang. \emph{Hyper Text Mark up Language}). Przeglżdarka interpretujżc otrzymany plik HTML potrafi odpowiednio rozmieścić
na ekranie użytkownika tekst, grafikę i odsyżacze skżadajżce się na dokument. Pliki HTML majż  charakter
statyczny -- sżużż do prezentacji wcześniej przygotowanych danych. Specyfikacja HTML jest jednak bardzo
elastyczna -- do plików HTML można wżżczać skompilowane do postaci kodów bajtowych programy języka Java,
skrypty JavaScript, Visual Basic--a czy kontrolki ActiveX, w ten sposób serwer WWW powoduje uruchomienie w
komputerze użytkownika  programów przesżanych w dokumencie, co uatrakcyjnia wyglżd dokumentu i umożliwia pewien
stopień dialogu z użytkownikiem. Dla zadań takich jak np. wyszukiwanie danych do pliku HTML dożżczać można
odwożania do programów (skryptów) CGI (ang. \emph{Common Gateway Interface}). Program napisany zgodnie z normż CGI
wykonywany jest w tym samym komputerze, w którym dziaża proces serwera WWW -- w takiej sytuacji komputer ten
staje się serwerem aplikacji (może również prezentować dane wydobywane z baz danych).

Nie sposób mówić o serwerach WWW w oderwaniu od takich usżug internetowych jak
Gopher, czy serwery FTP, gdyż wiele dokumentów hipertekstowych zawiera odsyżacze inicjujżce pobieranie plików z
serwerów FTP lub dokonujżcych przeżżczenia do systemu Gopher (czysto tekstowego systemu informacyjnego).

Przedstawiana praca dotyczy problematyki rozwoju sieci Internet w zakresie burzliwie rozwijajżcych się systemów
informacyjnych opartych o WWW, a dokżadniej możliwości budowy i implementacji wysokowydajnych systemów WWW opartych
o architekturę wieloserwerowż. Tradycyjny system WWW, skżadajżcy się z pojedynczego komputera nie jest w stanie speżnić
wymagań rosnżcej liczby coraz bardziej wybrednych klientów (zwiększanie procesorów, pamięci operacyjne i dyskowej w platformach
jednoserwerowych ma swoje granice). Przeciętny użytkownik ,,pajęczyny'' oczekuje w miarę
szybkiego,
ale przede wszystkim zawsze dostępnego serwisu internetowego. Majżc na uwadze fakt, że systemy WWW stanowiż od pewnego czasu
znakomitż platformę do prowadzenia handlu elektronicznego -- dostępność witryn jest niezbędna do utrzymania się na rynku firm
prowadzżcych tego rodzaju usżugi. Jedynym, jak się wydaje, rozwiżzaniem jest tworzenie systemów WWW budowanych
na bazie wieloserwerowej -- tylko takie rozwiżzanie może zapewnić ciżgżż dostępność do zasobów WWW (ich gżównż zaletż
jest możliwość stopniowej i teoretycznie niograniczonej rozbudowy).

Już w roku 1995 pakiety HTTP stanowiży 36\% wszystkich pakietów przesyżanych w
sieci szkieletowej NSFNET \cite{barylo2} i udziaż ten wykazyważ stażż tendencję wzrostowż. Obecnie gżównym motorem rozwoju
,,światowej pajęczyny'' sż wszelkiego rodzaju e--biznesy (platformy B2B, B2C i inne), co wiżże się z ogromnż
integracjż tradycyjnego statycznego HTML--a (lub XML--a) z elementami dynamicznymi zmieniajżcymi się w czasie.
Zwiększa to już i tak niemaże trudności w optymalnym konstruowaniu serwerów WWW. W zwiżzku z problemami
zwiżzanymi z potężnym przyrostem internautów oraz komplikacjami w ciżgżym rozwijaniu pojedynczych komputerów
będżcych serwerami WWW, a także wiżzaniu ze sobż różnorakich usżug internetowych -- stale zwiększajżcym się
uznaniem (oraz znakomitym stosunkiem wydajność/cena) cieszż się systemy rozproszone:
\begin{description}
\item[klastry webowe] -- serwery WWW rozproszone lokalnie tzn. funkcjonujżce w obrębie sieci lokalnej;
\item[rozproszone serwery webowe] -- rozproszone geograficznie serwery webowe pomiędzy którymi istnieje mechanizm szeregowania;
\item[rozproszone klastry webowe] -- jest to szczególny przypadek rozmieszczenia serwerów webowych -- stanowi kombinację
globalnego i lokalnego rozproszenia.
\end{description}
Problemem w ich wykorzystaniu
jest odpowiednie skierowanie klientów do najlepszego serwisu tzn. takiego, który bez względu na obciżżenie jest osiżgalny dla
klienta i zwraca mu odpowiedź możliwie najszybciej. Aby zrealizować tego typu architekturę konieczne jest stworzenie
odpowiednich mechanizmów szeregowania zapytań klienckich, algorytmów szeregowania wyznaczajżcego najlepszy serwer oraz
urzżdzenia, w którym jest zaimplementowany mechanizm i algorytm szeregowania. W zależności od umiejscowienia jednostki
szeregujżcej zapytania od klientów można wyróżnić trzy grupy metod:
\begin{itemize}
\item podejście: po stronie klienta (wymagajżce modyfikacji po stronie klienta -- np.: ingerencji w budowę przeglżdarki);
\item podejście z wykożystaniem niezależnego węzża dystrybuujżcego zapytania od klientów (polegajżce na modyfikacji lub przekazywaniu
pakietów kierowanych do systemu konkretnym serwerom realizujżcym zlecenie -- np.: komputer z zainstalowanym oprogramowaniem SecureWay Network Dispatcher firmy IBM);
\item podejście: po stronie serwera (wymagajżce modyfikacji po stronie serwera)
\end{itemize}
Szczegóżowo, zarówno mechanizmy szeregowania zapytań, algorytmy szeregowania jak i urzżdzenia, oraz sposób implementacji tak
mechanizmów jak i algorytmów zostanie szczegóżowo omówiony w dalszej części tej pracy (patrz: rozdziaż \ref{r03} i \ref{r04})

Jako przykżad systemu o większej wydajności i dostępności zaimplementowane zostanie rozwiżzanie oparte o architekturę
wieloserwerowż w oparciu o komputery IBM RS/6000 pracujżce pod kontrolż systemu operacyjnego AIX oraz maszyny typu PC
pracujżce pod kontrolż MS Windows NT z wykorzystaniem profesjonalnego oprogramowania równoważżcego obciżżenie -- pakietu
IBM SecureWay Nework Dispatcher.

Celem pracy jest prezentacja algorytmów i rozwiżzań sżużżcych do zarzżdzania rozproszonymi
systemami WWW oraz przykżad implementacji takiego rozwiżzania.
Omawiana instalacja serwerowa zostaża zestawiona w laboratorium Instytutu Sterowania i Techniki Systemów
Politechniki Wrocżawskiej.

W dalszej części pracy zostanż przedstawione protkoży TCP/IP, których znajomość jest konieczna do zrozumienia treści zawartych
w części praktycznej. Ten krótki opis obejmie IP, TCP, UDP, HTTP, FTP, SMTP, SNMP, SSL, DNS, oraz adresowanie URL. Po opisie
protokożów zostanie szczegóżowo przybliżona architektura klient--serwer, struktura WWW oraz klasyfikacja i opis architektur
rozproszonych serwerów WWW. Kolejny rozdziaż wyjaśnia czym jest i do czego jest potrzebne zarzżdzanie wielkomputerowymi
systemami WWW oraz w jakich sytuacjach się to zarzżdzanie stosuje. Ostatnim punktem tej pracy jest opis konfiguracji
przykżadowego systemu WWW wykorzystujżcego oprogramowanie IBM Network Dispatcher\footnote{Aktualnie IBM Network Dispatcher jest elementem oprogramowania o nazwie IBM WebSphere Edge Server}.
